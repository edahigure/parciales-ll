\documentclass[a4paper,10pt]{book}
\usepackage[utf8]{inputenc}

\usepackage[utf8]{inputenc}
\usepackage[english]{babel} %francais, polish, spanish, ...
\usepackage[T1]{fontenc}


\usepackage{amsfonts}
\usepackage{amsmath}
\usepackage{amssymb}
\usepackage{graphicx}
\usepackage{dcolumn}
\usepackage{latexsym}
\usepackage{hyperref}
\usepackage{bm}
\usepackage[english]{babel}
\usepackage{xcolor}



\usepackage{lmodern} %Type1-font for non-english texts and characters
\usepackage{amsthm}
\usepackage{amsfonts}
\usepackage{float}
\usepackage{ mathrsfs }


\usepackage[utf8]{inputenc}
\usepackage[T1]{fontenc}
\usepackage{textcomp}
\usepackage{gensymb}
\usepackage{subcaption}

\def \ep{\epsilon}


\begin{document}


\section{Introducci\'onwww}
Queremos buscar los m\'inimos del siguiente problema:$\mathscr{L}$ es un operador lineal
$\mathscr{L} u=f$.   Si definimos
\[A(u,v)=\int\int_{\Omega} \mathscr{L} u v  dx dy \] 
donde $dx dy$ representar\'a un diferencial de area y en general lo omitiremos.
$A$ es tal que $A(u,v)\geq \alpha {|| u ||}^2$ si  $u_0$ es m\'inimo de 
\[A(u,u)-2 (f,u) \]
donde se define
\[(f,u)=\int\int_{\Omega} f u .\]
entonces a $u_0$ se le llama la soluci\'on generalizada de $\mathscr{L} u=f$. El operador $\mathscr{L}$ puede ser 
por ejemplo la  biarm\'onica $\Delta^2$.
si
\[\int\int_{\Omega} \mathscr{L} u\phi =\int \int_{\Omega} f \phi  \]
para todo $\phi \in C_0^\infty$, entonces definimos el operador autoadjunto como 
\[\int\int_{\Omega} \mathscr{L}^{*} u\phi \]
donde $*$ indica complejo conjugado. Ejemplos f\'isicos de este operador son 
la membrana el\'astica y el campo electrost\'atico. En cualquiera de estos casos
vemos que la energ\'ia almacenada por el campo se puede escribir de la forma
\[I(u)=\int\int_{\Omega} |\nabla u |^2  .\]
El problema lo plantearemos como : encontrar $u$  que minimiza $I(u)$ con $u|_{\partial \Omega}=f$ .
Primero veremos que el minimizar la energia es equivalente a encontrar la soluci\'on al problema 
del laplaciano con valor en la frontera. Sea $v$ tal que $v|_{\partial \Omega}=f$ y sea
$h=v-u$ entonces $h|_{\partial \Omega}=0$. Ahora
\[I(v)= I(u+h)=  \int \int_{\Omega} (u_x +h_x)^2+(u_y+h_y)^2  \]
donde si utilizamos una de las identidades de Green que se sigue del teorema de la divergencia con 
el campo vectorial $u \nabla v $ y usando el hecho que $\partial_{k}(u\partial_k v)=u \partial _k \partial _k v+  \partial _k u \partial _k v$,
donde se asume  suma sobre indices repetidos, entonces

\[\int\int_{\Omega} \nabla \cdot (u \nabla v)   =\int\int u \Delta v + \int\int_\Omega \nabla u\cdot\nabla v \]
donde  aplicandolo a la energ'ia y recordando que $h$ vale cero en la frontera encontramos que

\[I(v)=I(u)+I(h)-2\int\int_{\Omega} h \Delta u  \] 
donde usamos el hecho que $h$ vale cero en la frontera. Si el laplaciano de $u$ es positivo en un punto $(x_0,y_0)$
entonces ser\'a positivo en una vecindad de $(x_0,y_0)$.  Tomamos $h<0$ en la vecindad de $(x_0,y_0)$
fuera de dicha vecindad, entonces se tiene que $I(u+h)\geq I(u)$. Ahora si tomamos
$h>0$ en la vecindad y cero fuera y tal que $\int\int_{\Omega} |\nabla h|^2 =1$   entonces
\[I(u+\epsilon h)-I(u)= \epsilon^2-2\epsilon \int\int_{\Omega} h \Delta u <0\]
si $ \epsilon$ es peque\~no, lo que contradice que $I(u)$ sea m\'inimo. Lo que implica que
$\Delta u\leq 0$ y por lo tanto $\Delta u=0$ y por lo tanto $u$ m\'inimo de $I(u)$  con $u|_{\partial \Omega} =f$ es necesario
y suficiente para que $\Delta u=0$ en $\Omega$ y $u|_{\partial \Omega}=f$.

\subsection*{\underline{La soluci\'on es \'unica}.}

Si $u_1$ y $u_2$ son soluciones, $h=u_1-u_2$ entonces
\[I(u_1)=I(u_2+h)=I(u_2)+I(h)\]
y como $I(u_1)=I(u_2)$ implica que $I(h)=0$ y por lo tanto $\nabla h=0$ y por lo tanto 
$h=cte=0$ en $\partial \Omega$.

\subsection*{\underline{Principio de Dirichlet}.}
Tenemos las siguientes preguntas:
\begin{enumerate}
 \item \textquestiondown Existe una soluci\'on de $Delta u=0$ 
en $\Omega$, talque $u|_{\partial \Omega}=f$ ?.\textquestiondown Que tipo de soluci'on es ?
Acaso $u\in C^2 (\Omega) \cap C^0 (\bar{\Omega})$ donde la barra indica la cerradura de $\Omega$.
Si es \'unica  $h=u_1-u_2$ implica $\Delta h=0$ y $h|_{\partial \Omega} =0$ implica el principio
del m\'aximo que a su vez implica que $h=0$.

\item \textquestiondown Existe un m\'inimo ?
\item \textquestiondown Como se calcula el m\'inimo $I(u)=\int \int_{\Omega}  |\nabla u|^2 $.
\item Si existe el m\'inimo $u$ implica acaso que $u\in C^2 (\Omega) \cap C^0 (\bar{\Omega})$ implica 
acaso que la energ\'ia es finita ?
\end{enumerate}

Ilustraremos estas preguntas con un ejemplo. 

\subsection*{1)\underline{Ejemplo de Hadamard}.}

Sea $\Omega $ un disco unitario. Consideremos la funci\'on sobre la frontera del 
disco unitario parametrizada por el \'angulo $\theta\in [0,2\pi]$

\begin{equation}
\label{hadamard}
  f(\theta)=\sum_{n=1}^{\infty} \frac{cos(2^{2n} \theta)}{2^n}
\end{equation}

Como $|f(\theta)|\leq \sum \frac{1}{2^n} =1$ (como puede verse de sumar la serie geom\'etrica con $r=1/2$),
entonces al serie es absolutamente convergente. Que implica que $f(\theta)$ es continua.
sea 
\[u(x,y)=u(\rho,\theta)=\sum_{n=1}^{\infty}  \rho ^{2 ^{2n} } \frac{\cos(2^{2n} \theta)}{2^n}\]
esta funci\'on cumple que $u|_{\partial \Omega}=f(\theta)$  y $|u(x,y)|\leq 1$ y
\[u(x,y)= Re \left( \sum_{n=1}^\infty \frac{z^{2^{n}} }{2^n}\right) \]
es una funci\'on anal\'itica para $|z|<1$  lo que implica que $u$ es arm\'onica. Ahora para pasar a coordenadas
polares usamos el hecho que 
\[u_x=u_\rho \cos \theta - u_{\theta} \frac{\sin \theta}{\rho} \]
\[u_y=u_\rho \sin \theta  + u_{\theta} \frac{\cos \theta}{\rho}.\]

La energ\'ia se calcula como
\[ I(u)= \int\int_{\Omega} (u_x^1+u_y^2)= \int_0^1 \int_0^{2\pi} \left( u_{\rho} ^{2} + \frac{u_\theta ^2 }{\rho^2} \right) \rho d\rho d\theta
\]
Sustituyendo se tendr\'ia que

\[ I(u)= \int_0^1 \int_0^{2\pi} \left( 
 \sum_{n=1}^{\infty}  2^{2n} \frac{\rho ^{2 ^{2n}-1 } }{2^n} \frac{\cos(2^{2n} \theta)}{2^n} \right)^2+
\]
\[
 \left( 
 \sum_{n=1}^{\infty} (-) 2^{2n} \frac{\rho ^{2 ^{2n} } }{2^n} \frac{\sin(2^{2n} \theta)}{2^n} \right)^2 \frac{1}{\rho^2} \rho 
d\rho d\theta.
\]

Como los dobles productos de senos y cosenos son cero, entonces

\[I(u)=\left.2\pi \sum_{n=1}^{\infty} \frac{2^{2n}}{2^{2n+1} } \rho^{2^{2n+1}}\right|_0^1=\pi \sum_{n=1}^{\infty} \rho^{2n+1}=\infty.\]

Moraleja, tengo que trabajar con una clase m\'as chica de funciones en la frontera. Esta clase ser\'a $f|_{\partial \Omega}$ y
si definimos los siguientes espacios, entonces

\[ L^2 (S^1)=\left\lbrace g(\theta)=\sum_{n=-\infty}^{\infty} a_n e^{i n \theta } \in Re \Leftrightarrow
 a_{-n}=\bar{a}_n \right\rbrace
\]
con 
\[\sum |a_n|^2 =\int_{0}^{2\pi} |g(\theta)|^2 d\theta < \infty \]

Podemos definir los espacios de funciones
\[H^{\alpha}(S^1)=\left\lbrace  g(\theta) = \sum_{-\infty}^{\infty} a_n e^{in\theta}, a_{-n}=\bar{a}_n, \sum_{-\infty}^{\infty} |a_n|^2 n^{2\alpha} <\infty  \right\rbrace.
\]
en particular
\[H^1(S^1)=\left\lbrace g: \sum_{n=-\infty}^{\infty} | a_n|^2 n^2 =\int_0^{2\pi} |g'(\theta)|^2 d\theta \right\rbrace < \infty \] 
Para ver a que espacio de funciones corresponde la funci\'on  del ejemplo de Hadamard (\ref{hadamard}) vemos que
\[\sum_{-\infty}^{\infty} |a_n|^2 n^{2\alpha} =\sum_{n=-\infty}^{\infty} \left(\frac{1}{2^n} \right)^2 (2^{2n} )^{2\alpha} =\sum_{n=-\infty}^{\infty} \frac{1}{2^{2n(1-2\alpha)}} \] 
que ser\'a menor que infinito si $\alpha <1/2$  y la serie converge uniformenente. Si $\alpha\geq 1/2$ tenemos una
suma infinita y por lo tanto $f(\theta)\in H^{\alpha}(S^{1} )$ con $\alpha<1/2$ .
Se puede probar que si $f(\theta)\in H^{\alpha}(S^1)$, $\alpha\geq 1/2$ entonces existe
$u$ tal que $u|_{\rho =1}=f$ y $I(u)<\infty$ . (teorema de traza, adem\'as $f$ es continua)

\subsection*{2)\underline{Tipo de soluciones}.}

Sea $f|_{\partial \Omega}$ tal que exista $u_0 \in C^2(\Omega) \cap  C^0(\bar{\Omega}) $, $u_0|_{\partial \Omega}=1$ y $I(u_0)< \infty$.
Sea $u$ talque $\nabla^2 u=0$ , $u\in C^2(\Omega)\cap C^0(\bar{\Omega})$, $I(u)<\infty$, $u|_{\partial \Omega}=f$ sea 
$h=u-u_0$ implica $h \in C^2(\Omega) \cap  C^0(\bar{\Omega})$, $\Delta h=\Delta u-\Delta u_0=-\Delta u_0=g$  es continua. 

Sea 
\[J(k)=\int\int_\Omega| \nabla k|^2  + 2\int \int_\Omega gk \]
para $k\in C^2 (\Omega)\cap C^0(\bar{\Omega}) $, $k|_{\partial \Omega}=0$ , $I(k)<\infty$. Si definimos

\[J(h)=I(h)+2 \int\int_\Omega \Delta h h  = I(h) -2\int \int_\Omega | \nabla h |^2 = -I(h)\]
donde hemos utilizado la identidad de Green. Ahora 
\[J(h+k)= I(h+k)+2\int \int (h+k)g  \]
\[J(h+k)= I(h)+I(k) -2 \int \int k \Delta h+2\int \int  (h+k)g \]
\[J(h+k)=J(h) +I(k)+2 \int\int_\Omega( g-\Delta h )k \]
cuando $g-\Delta h =0$ da el m\'inimo  de $J(k)$. Alrev\'es,  si $h$ tiene esas propiedades y $J(h)$ es
m\'inimo entonces $\Delta h=g$ en $\Omega$. Entonces tenemos que si definimos el espacio de funciones
\[E=\left\lbrace k\in C^2 (\Omega) \cap C^0(\bar{\Omega} ), k|_{\partial \Omega}=0 \right\rbrace, I(k)<\infty\]
entonces $  \underset{E}{min} J(k)$  es una condici\'on necesaria y suficiente para que $h$ 
sea soluci\'on de la ecuaci\'on si $\Delta h=g$ y como se vi\'o se tiene que

\begin{enumerate}
\item $J(h)=-I(h)$
\item $J(h+k)= J(k)+I(k)$
\item $J(k)\geq I(k)-2\int\int_\Omega |g||k| \geq I(k) - \epsilon \int\int_\Omega k^2 -\frac{1}{\epsilon} \int\int_\Omega g^2 $
\end{enumerate}
donde en $3)$ se utiliz\'o el hecho que si $a,b$ son dos n\'umeros cualesquiera entonces siempre se tiene
que  $2ab\leq \epsilon a^2 +\frac{1}{\epsilon} b^2 $ (teorema del binomio) . Si $|k|=0$ en $\partial\Omega$ entonces el lema de Poincar\'e
nos da que 
\[ \int\int_\Omega k^2 \leq \kappa \int\int_\Omega |\nabla k|^2\]
donde tomando  $\epsilon=1/\kappa$ tenemos que
\[J(k)\geq -\kappa \int\int_\Omega g^2 \]
es decir $J$ es acotado por abajo.


\chapter{Espacios de Hilbert}

\section{Espacios vectoriales}
Sea $E$ con operaciones $+$ y $\times$  por escalar, $f,g \in E$  $\lambda\in \mathbb{R}$(ó $\mathbb{C}$),
entonces   $\lambda f \in E$.
Ejemplos:  $\mathbb{R}$, $\mathbb{C}$  $C^0(\bar{\Omega}),C^1(\bar{\Omega}),...$
\underline{Espacio Normado} : E vectorial con $\Arrowvert \ \Arrowvert:E\rightarrow \mathbb{R}^{+}$
$\Rightarrow$  $\lVert f-g \rVert$ = distancia de $f$ a $g$\\
$\lVert f \rVert =0  \Leftrightarrow f = 0 $ \\
$\lVert \alpha f \rVert =  | \alpha |\lVert f \rVert$ \\ 
$\lVert  f + g  \rVert =  \lVert f \rVert +\lVert g  \rVert$ \\ 
Ejemplos: $\mathbb{R}^n$ con la norma $\lVert x \rVert= \underset{j=1,... n}{\max} |x_j|$ , o, $\lVert x\rVert=\left( \sum_j x_j^ 2 \right) ^{1/2}$\\
$C^0(\bar{\Omega})$ , $\lVert f \rVert = |f|_0 = \underset{x\in \bar{\Omega} }{\max} | f(x)|$\\
$C^1(\bar{\Omega})$ , $\lVert f \rVert = |f|_1 = \underset{x\in \bar{\Omega} }{\max} | f(x)|$ + 
$\underset{\lambda}{\sum} \underset{x\in\Omega }{\max} \left| \partial_{\lambda} f \right|$ \\
$L^2(\Omega)$ con la norma $\lVert f \rVert_2= \left( \int \int_\Omega |f|^2 dx \right)^\frac{1}{2}$\\ 
$H^1(\Omega)$ con la norma $\lVert f \rVert_1= \left( \int \int_\Omega |f|^2 dx + \underset{\lambda}{\sum}  \left| \partial_{\lambda} f \right| \right)^\frac{1}{2}$\\
$L^2(\Omega) = \{ f \ \ medible\ \ con \ \ \int \int_\Omega |f|^2 dx< \infty \} $\\  

\underline{Espacio con producto escalar}
Espacio vectorial con $( \ \ , \ \ )  \in \mathbb{R}$ (ó $\mathbb{C}$). Si $(f,g)=\overline{(g,f)}$   
entonces $(f,f)\in \mathbb{R}$. \\

$(f_1+f_2,g)=(f_1,g)+(f_2,g)$  \\
$(f,f)>0$ si $f\neq 0$ entonces se define $\lVert f \rVert = (f,f)^{\frac{1}{2}}$ porque
de la desigualdad de Cauchy-Schwarz se tiene
$|(f,g)| \leq \lVert f \rVert \lVert g\rVert$ implica \\

\[\lVert f+g \rVert^2 = (f+g,f+g) =\lVert f\rVert^2 + \lVert g\rVert^2 + (g,f)+(f,g)\]
\[ \leq \lVert f\rVert^2 + \lVert g\rVert^2 + \lVert f \rVert\lVert g \rVert= \left( \lVert f\rVert + \lVert g\rVert  \right)^2 \]

\underline{Cauchy-Schwarz} \\
$0\leq (f+\lambda g, f + \lambda g) = \lVert f \rVert^2 +\lambda^2\lVert g \rVert^2 + \bar{\lambda} (f,g)+ \lambda (g,f)= \lVert f \rVert^2+ \lambda^2\lVert g \rVert^2 +2 Re (\lambda(g,f))$
si $(g,f)= R e^{i \phi} $ tomemos $\lambda= \rho e^{-i\phi}$  entonces\\
$0\leq \rho^2 \lVert g\rVert +2 \rho | (g,f) | + \lVert f\rVert^2 $
no tiene raíces implica el discriminante \\
$ (g,f) |^2 - \lVert f \rVert+ \lambda^2\lVert g \rVert \leq 0$\\
\underline{ejemplos} \\
$\mathbb{R}^n$,$\mathbb{C}$ $(x,y)= \sum x_i \bar{y}_i$ \\
$L^2(\Omega) : (f,g) =\int_{\Omega} f \bar{g} dx$\\
$H^1(\Omega)$ con producto interno $(f,g)_1 = \int_{\Omega} f\bar{g} + \sum_{i=1}^{n} f_{x_i} \bar{g}_{x_i}$

\section{Sucesión de Cauchy en un espacio normado}
Sea $E$ con $ \lVert \ \ \rVert$, Una sucesión $\{ u_n\}\in E$ es de Cauchy si dado $\epsilon>0$
existe $N(\epsilon)$ tal que $\lVert u_n -u_m\rVert \leq \ep $ , $n,m\geq N$ .

$E$ es de Banach si $E$ tiene $(\ ,\ )$ y es completo, $E$ es de Hilbert.

No todo espacio vectorial es completo, pero se puede completar tomando clases de equivalencia de sucesiones
de Cauchy por ejemplo $C^0[0,1]$    

\end{document}